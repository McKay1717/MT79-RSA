\documentclass[a4paper,10pt]{article}
\usepackage[utf8]{inputenc}
\usepackage{amsmath}
\usepackage{graphicx, listings, hyperref}
\lstloadlanguages{Ruby}

% Title Page
\title{Cryptographie : le système RSA}
\author{Aurélien Blais, Nicolas Iung}


\begin{document}

\maketitle
\clearpage

\section{Principe du chiffrement RSA}
\subsection{Question 1}
\subsubsection{Montre que $cd = 1 + k(p-1)(q-1)$}

En utilisant le théorème de Bezout $au+bv=1$\\
On pose
\begin{align*}
a &= c\\
u &= d\\
b &= \varphi(n)\\
v &= -k
\end{align*}

\begin{align*}
cd + -k\varphi(n) &= 1\\
cd &= 1 + k\varphi(n)\\
\text{Or } \varphi(n) &= (p-1)(q-1)\\
cd &= 1 + k(p-1)(q-1)
\end{align*}

\subsection{Question 2}
\subsubsection{Déduire que $cd \equiv 1 \pmod{\varphi(n)}$}

On sait que
\begin{align*}
a &\equiv b \pmod c\\
\iff a &= b + kc
\end{align*}

On peut donc en déduire que
\begin{align*}
cd &= 1 + k(p-1)(q-1)\\
cd &= 1 + k\varphi(n)\\
cd &\equiv 1 \pmod{\varphi(n)}
\end{align*}

\subsubsection{Conclure que si $c$ et $\varphi(n)$ premiers entre eux, il existe toujours un entier $d$ inverse de $c$ modulo $\varphi(n)$}

\begin{align*}
&\text{On a } cd \equiv 1 \pmod{\varphi(n)}\\
&\text{Or on dit que $a$ est l'inverse de $b \equiv \pmod n$}\\
&\text{Si et seulement si $ab \equiv 1 \pmod n$}\\
&\text{On a donc $d$ inverse de $c \pmod{\varphi(n)}$}
\end{align*}

\subsection{Question 3}
\subsubsection{Montrer que $M$ est premier avec $p$ et avec $q$}
$M$ est un entier qui représente le message coder. On montrera que $M$ est premier avec $p$ et qu’il est aussi avec $q$.
Sachant que $p$ et $q$ sont des nombres premiers.
Pour cela on admet que 1 et $p$ divise $q$ ou 1 et $q$ divise $p$. Alors respectivement $M$ est premier avec $p$ et $q$. Car $M$ est un message codé en un entier inférieur à $n$ avec $n = p*q$.


\subsection{Question 4}
\subsubsection{Déduire que $M^{p-1} \equiv 1 \mod(p)$}
Notons que $p$ est un nombre premier.
D'après le théorème de Fermat, Si $M$ n’est pas divisible par $p$ alors $M^{p-1} \equiv 1 \mod(p)$.\\
Or $M$ et $p$ sont premier entre eux.
Donc le théorème de Fermat s'applique et $M^{p-1} \equiv 1 \mod(p)$

\subsubsection{Déduire que $M^{q-1} \equiv 1 \mod(q)$}
Notons que $pq$ est un nombre premier.
D'après le théorème de Fermat, Si $M$ n’est pas divisible par $p$ alors $M^{q-1} \equiv 1 \mod(p)$.\\
Or $M$ et $q$ sont premier entre eux.
Donc le théorème de Fermat s'applique et $M^{q-1} \equiv 1 \mod(p)$

\subsection{Question 5}
\subsubsection{Déduire que $M^{cd}\equiv M \mod(p)$}
Rappelons que :
\begin{equation}
	cd = 1+k(p-1)(q-1)
\end{equation}
En appliquant cela à M :\\
\begin{equation}
 M^{cd}=M^{1+k(p-1)(q-1)}=M*(M^{p-1})^{k(q-1)}\equiv M \mod(p)
\end{equation}
Donc $p$ divise $M^{cd}-M$
\subsubsection{Déduire que $M^{cd}\equiv M \mod(q)$}
De même que pour$M^{cd}\equiv M \mod(p)$, $M^{cd}\equiv M \mod(q)$ c'est à dire que que $q$ divise $M^{cd}-M$ 
\subsection{Question 6}
\subsubsection{Déduire que $M^{cd} - M$ est un multiple de n}
$p$ et $q$ sont premier entre eux, $pq=n$ divise $M^{cd} - M$.\\
Donc $M^{cd}\equiv M \mod(n)$
\clearpage
\section{Premier exemple}
\subsection{Question 1}
\subsubsection{Calculer $n_1 = p_1 q_1$ et $\varphi(n_1) = (p_1 - 1)(q_1 - 1)$ avec $p_1 = 7307$ et $q_1 = 5923$}

\begin{align*}
n_1 &= p_1 q_1\\
n_1 &= 7307 * 5923\\
n_1 &= 43279361
\end{align*}

\begin{align*}
\varphi(n_1) &= (p_1 - 1)(q_1 - 1)\\
\varphi(n_1) &= (7307 - 1)(5923 - 1)\\
\varphi(n_1) &= 43266132
\end{align*}

\subsection{Question 2}
\subsubsection{Choisir un entier $c_1$ premier avec $\varphi(n_1)$ tel que $c_1 < \varphi(n_1)$}
On sait que 2 nombres sont premiers entre eux si leur $PGCD$ est égal à 1\\
On pose donc $c_1 = 5$
\begin{align*}
\text{Par la Méthode d'Euclide}&\\
43266132 &= 8653226*5 + 2\\
5 &= 2*2 + 1\\
2 &= 2*1 + 0
\end{align*}
Le PGCD est égal au dernier reste non nul soit 1.\\ 
Le PGCD étant égal à 1, $c_1 = 5$ et $\varphi(n_1) = 43266132$ sont premiers entre eux et $c_1 < \varphi(n_1)$.

\subsection{Question 3}
\subsubsection{Déterminer $d_1$ inverse modulaire de $c_1$ modulo $\varphi(n_1)$}

\clearpage
\section{Fonctions de base}
\subsection{Question 1}
\subsubsection{Implémenter \textit{exponentiationModulaire(x, k, n)}}
L'algorithme implémenté suit le pseudo-code suivant :\\
 \url{https://en.wikipedia.org/wiki/Modular_exponentiation#Right-to-left_binary_method}\\
 Ayant une complexité en $O(log(k))$
\begin{lstlisting}[language=Ruby]
  def self.exponentiation_modulaire(x, k, n)
    result = 1
    while k > 0
      result = (result * x) % n if (k & 1) == 1
      k    = k >> 1
      x = (x**2) % n
    end
    result
  end
\end{lstlisting}

\subsection{Question 2}
\subsubsection{Implémenter \textit{euclideEtendu(a, b)}}
L'algorithme implémenté suit le pseudo-code suivant :\\
 \url{https://fr.wikipedia.org/wiki/Algorithme_d%27Euclide_%C3%A9tendu#L'algorithme}\\
 \begin{lstlisting}[language=Ruby]
   def self.euclide_etendu(a, b)
    r, u, v, r2, u2, v2, q = a, 1, 0, b, 0, 1, 0
    while(r2 > 0) do
      q = r/r2
      r, u, v, r2, u2, v2 = r2, u2, v2, r-q*r2, u-q*u2, v-q*v2
    end
    {pgcd: r, u: u, v: v}
  end
\end{lstlisting}
La fonction retourne un $Hash$, c'est à dire ensemble clé $\Rightarrow$ valeur\\
Tel que \{$pgcd: valeur, u: valeur, v: valeur$\}

\clearpage
\subsection{Question 3}
\subsubsection{Implémenter \textit{inverseModulaire(a, N)}}
La méthode retourne l'inverse modulaire de $(a, N)$ et se base sur la définition fournie ici :\\
\url{https://fr.wikipedia.org/wiki/Inverse_modulaire#Algorithme_d'Euclide_%C3%A9tendu}\\
Si a et N ne sont pas premiers entre eux, la méthode lève une exception.
\begin{lstlisting}[language=Ruby]
  def self.inverse_modulaire(a, n)
    val = euclide_etendu a, n
    raise Exception.new("Can't find value for a: #{a} and n: #{n}") unless val[:pgcd] == 1
    val[:u] % n
  end
\end{lstlisting}

\subsection{Question 4}
\subsubsection{Implémenter \textit{generationExposants(p, q)}}
On déclare $\varphi = (p-1)*(q-1)$ comme vu précédemment dans l'énoncé.\\
On déclare ensuite $c = 2$, afin de ne pas obtenir $c=1$, qui est premier avec l'ensemble des entiers.\\
Pour obtenir $c$, on l'incrémente tant que $PGCD(c, \varphi)$ n'est pas égal à 1.\\
Enfin, on retourne un $Hash$ contenant la valeur de $c$ et de $d$ tel que $d = inverseModulaire(c, \varphi)$
\begin{lstlisting}[language=Ruby]
  def self.generation_exposants(p, q)
    phi = (p - 1) * (q - 1)
    c = 2
    while c < phi
      break if c.gcd(phi) == 1
      c += 1
    end
    {c: c, d: inverse_modulaire(c, phi)}
  end
\end{lstlisting}

\clearpage
\subsection{Question 5}
\subsubsection{Implémenter \textit{chiffrement(m, n, c)}}
On sait d'après l'énoncé que $m_2 \equiv m^c \mod(n)$. L'algorithme renvoie donc cette valeur, qui est l'exponentiation modulaire.\\
\begin{lstlisting}[language=Ruby]
  def self.chiffrement(m, n, c)
    exponentiation_modulaire m, c, n
  end
\end{lstlisting}

\subsubsection{Implémenter \textit{dechiffrement(m, n, d)}}
On sait d'après l'énoncé que $m \equiv m_2^c \mod(n)$. L'algorithme renvoie donc cette valeur, qui est l'exponentiation modulaire.\\
\begin{lstlisting}[language=Ruby]
  def self.dechiffrement(m, n, d)
    exponentiation_modulaire m, d, n
  end
\end{lstlisting}

\clearpage
\section{Chiffrement de messages textes}
\subsection{Un mot, un nombre}
\subsubsection{Méthode \textit{StringToInteger}}
\subsubsection{Méthode \textit{IntegerToString}}

\clearpage
\subsection{A vous de jouer}
\subsubsection{Ré-implémentation de la méthode \textit{StringToInteger} en Ruby}
\begin{lstlisting}[language=Ruby]
  def self.string_to_integer(message)
    message = message.upcase.split(//)
    value = 0

    message.each_with_index do |char, i|
      value = value + (ALPHABET.length ** (message.length - 1 - i)) * ALPHABET.index(char)
    end
    value
  end
\end{lstlisting}
La méthode est une copie de celle fournie en Java, l'alphabet est lui aussi repris à l'identique.\\
\begin{lstlisting}[language=Ruby]
  ALPHABET = %w(. A B C D E F G H I J K L M N O P Q R S T U V W X Y Z).freeze
\end{lstlisting}
\bigskip

\subsubsection{Ré-implémentation de la méthode \textit{IntegerToString} en Ruby}
\begin{lstlisting}[language=Ruby]
  def self.integer_to_string(number)
    quotient = number / ALPHABET.length
    remainder = number % ALPHABET.length
    message = "#{ALPHABET[remainder]}"

    while quotient > ALPHABET.length
      remainder = quotient % ALPHABET.length
      quotient = quotient / ALPHABET.length
      message += ALPHABET[remainder]
    end
    
    (message + ALPHABET[quotient]).reverse
  end
\end{lstlisting}
La méthode est une copie de celle fournie en Java.\\
Exception faite que l'on construit le mot à l'envers, et qu'il est donc inversé avant d'être renvoyé. 

\clearpage
\subsubsection{Implémentation de la méthode \textit{Decodage(message, n, c)}}
\begin{lstlisting}[language=Ruby]
  def self.decode(message, n, c)
    prime_div = n.prime_division
    p = prime_div[0][0]
    q = prime_div[1][0]

    d = RSA.inverse_modulaire(c, (p - 1) * (q - 1))

    message = RSA.string_to_integer message

    RSA.integer_to_string RSA.dechiffrement message, n, d
  end
\end{lstlisting}
On commence par déterminer $p$ et $q$, pour cela on utilise la méthode $prime\_division$ qui renvoie les facteurs premiers d'un nombre donné.\\
On peut ainsi calculer $\varphi(n)$ et donc déterminer $d$ en utilisant la méthode $inverse\_modulaire(c, \varphi(n))$.\\
Il ne reste plus qu'a déchiffrer le message, en utilisant la fonction $dechiffrement(message, n, d)$

\subsubsection{Déchiffrer les valeurs données}
On execute notre méthode $decode(message, n, c)$ avec les valeurs fournies.
\begin{lstlisting}[language=Ruby]
puts RSA.decode "LHRZNS", 211582871, 127		# => "RETI"
puts RSA.decode "AYMRNCI", 844991843, 349837		# => "RUIZ"
puts RSA.decode "IVWTRM.FPL", 202899206548601, 39898535	# => "SICILIENNE"
\end{lstlisting}
Les valeurs trouvées correspondent bien a des ouvertures d'échecs.

\clearpage
\section{Pour aller plus loin}
\subsection{Utilisation de RSA en pratique}
\subsection{Générer et vérifier de grands nombres premiers}
\subsection{Algorithme de Shor et fiabilité et sécurité de RSA}
\end{document}
